\documentclass{article} % El documento es de tipo artículo
\usepackage[utf8]{inputenc} % Paquete que permite escribir caracteres especiales
\usepackage[spanish,mexico,es-tabla]{babel} % Paquete para cambiar "Cuadro" a "Tabla" en encabezados de tablas
\usepackage{graphicx} % Paquete para importar figuras
\usepackage{hyperref} % Paquete para agregar vínculos como enlaces
\usepackage{booktabs}
\usepackage{authblk}
\usepackage{amsmath}
\usepackage{csvsimple}
\usepackage{pythontex}
\usepackage{xcolor}
\usepackage{float}
\usepackage{caption}
\usepackage[a4paper,top=2cm,bottom=2cm,left=3cm,right=3cm,marginparwidth=1.75cm]{geometry} 

\author{Fernando Alvarez y Maritza Bello}

\title{Tendencia poblacional cormorán orejón \\ \begin{large} Grupo de Ecología y Conservación de Islas \end{large}}

\begin{pycode}
import json

json_alcatraz_path = 'non-tabular/jsonLambdaCormorantAlcatraz.json'
with open(json_alcatraz_path, encoding='utf8') as results_file:
    results_alcatraz = json.load(results_file)

json_patos_path = 'non-tabular/jsonLambdaCormorantPatos.json'
with open(json_patos_path, encoding='utf8') as results_file:
    results_patos = json.load(results_file)

json_pajaros_path = 'non-tabular/jsonLambdaCormorantPajaros.json'
with open(json_pajaros_path, encoding='utf8') as results_file:
    results_pajaros = json.load(results_file)

\end{pycode}

\begin{document}

\maketitle

\begin{abstract}
Calculamos la tasa de crecimiento para cormorán orejón en Isla Alcatraz, Patos y Pajaros, se encontró que las poblaciones han disminuido ligeramente ($\lambda < 1$).
\end{abstract}

\section*{Metodología}
\subsection*{Modelo}
A continuación se muestra el modelo del tamaño poblacional:

\begin{equation}
N(t)=N_{0}\lambda^{t}
\end{equation}

\noindent donde $N_{0}$ representa la población inicial, $\lambda$ la tasa de crecimiento fundamental y $t$ el intervalo de tiempo. 

\subsection*{Cálculo de tasa de crecimiento}

Utilizamos la base de datos \href{https://drive.google.com/drive/folders/1aXmotwcGcZjK52USWMdlZoffaMUlI0tT}{{\color{blue}\textit{\underline{conteo\_nidos\_cormoran}}}}, donde tenemos datos de cantidad de nidos activos para diferentes temporadas durante los últimos 20 años. Para calcular la tasa de crecimiento fundamental $\lambda$, tomamos los máximos de cada temporada. Consideramos una temporada como el intervalo de tiempo entre los meses de septiembre hasta abril.

\subsection*{Bootstrapping no paramétrico}

Hicimos un remuestreo no paramétrico de los datos de cantidad de madrigueras. Este procedimiento no hace suposiciones de la distribución de los datos \cite{carpenter2000bootstrap}. El algoritmo que implementamos es el siguiente: 

\begin{itemize}
    \item Tomamos una muestra de $n$ observaciones aleatorias con la posibilidad de remplazo, a partir del conjunto original de datos de cantidad de madrigueras con actividad, $ N_{\mbox{obs}} $. Esta muestra es un conjunto de datos bootstrap, $N^{*}$. Donde $n$ es el total de registros ($n=14$ para Alcatraz, $n=5$ para Patos y $n=6$ para Pajaros).
    \item Calculamos la versión bootstrap $\lambda^*$ de la tasa de crecimiento $\lambda$, ajustando el modelo (1) al nuevo conjunto $N^{*}$.
    \item Repetimos los pasos anteriores $B = 2000$ veces y obtenemos la distribución bootstrap de la tasa de crecimiento $\lambda^*$.
\end{itemize}

Con esto, calculamos un intervalo de confianza del 95\% para $ \lambda^* $. 

\section*{Resultados}

En la Tabla 1 mostramos las tasas de crecimiento $\lambda^*$ y su respectivo intervalo de confianza del 95\%. Por ejemplo para isla Patos, el valor central de la distribución es $0.76$, el limite inferior es $0.76 - 0.14 = 0.62$ y el limite superior $ 0.76 + 0.15 = 0.91$. A continuación podemos ver una gráfica para los datos de cada isla y el modelo ajustado.


\begin{table}[h]
\centering
\caption{Se muestran las tasas de crecimiento $\lambda^*$ y sus intervalos de confianza del 95\%, acompañados del p-valor asociado a la hipótesis de que la población está disminuyendo. Calculadas a partir de un ajuste del modelo (1) a los datos del máximo número de nidos en cada temporada y un remuestreo bootstrap.}
\renewcommand{\arraystretch}{1.3}
\begin{tabular}{ccc}
\hline
Isla & $\lambda$ & p-valor \\ 
\hline
Alcatraz & $ \py{'%.2f'%(float(results_alcatraz['lambda'][0]))}_{-\py{'%.2f'%(float(results_alcatraz['limite_inferior'][0]))}}^{+ \py{'%.2f'%(float(results_alcatraz['limite_superior'][0]))}} $ & \py{'%.2f'%(float(results_alcatraz['p-valor_menor'][0]))} \\ 
Patos &  $ \py{'%.2f'%(float(results_patos['lambda'][0]))}_{-\py{'%.2f'%(float(results_patos['limite_inferior'][0]))}}^{+ \py{'%.2f'%(float(results_patos['limite_superior'][0]))}} $ & \py{'%.2f'%(float(results_patos['p-valor_menor'][0]))}\\ 
Pajaros & $ \py{'%.2f'%(float(results_pajaros['lambda'][0]))}_{-\py{'%.2f'%(float(results_pajaros['limite_inferior'][0]))}}^{+ \py{'%.2f'%(float(results_pajaros['limite_superior'][0]))}} $ & \py{'%.2f'%(float(results_pajaros['p-valor_menor'][0]))} \\
\hline
\end{tabular}
\end{table}

El valor del p-valor para aceptar o rechazar una hipótesis es de $0.1$. El p-valor para isla Alcatraz y Pajaros es $0.01$ y en isla Patos el p-valor fue $0.07$. Esto nos dice que podemos aceptar la hipótesis de que la población está realmente disminuyendo. Podemos entonces concluir que los datos, nos brindan información suficiente para decir con seguridad que la poblacion está disminuyendo en las tres islas.

\subsection*{Isla Alcatraz}

\begin{figure}[H]
\hspace{-2cm}
    \includegraphics[scale=0.7]{figures/pngPopulationGrowRateCormorantAlcatraz.png}
\caption{En azul se muestran los máximos de la cantidad de nidos activos de cormorán orejón para cada temporada (Sep-Abr) de los ultimos 20 años en Isla Alcatraz. En naranja el modelo de crecimiento poblacional ajustado.}
\end{figure}

Descartamos el punto correspondiente a la temporada 2005-2006. El tamaño poblacional oscila entre 800 y 2000 parejas. 

\subsection*{Isla Patos}

\begin{figure}[H]
\hspace{-2cm}
    \includegraphics[scale=0.7]{figures/pngPopulationGrowRateCormorantPatos.png}
\caption{En azul se muestran los máximos de la cantidad de nidos activos de cormorán orejón para cada temporada (Sep-Abr) del periodo 2012-2018 en Isla Patos en Sinaloa. En naranja el modelo de crecimiento poblacional ajustado.}
\end{figure}

Descartamos los puntos correspondientes a las temporadas 2011-2012, 2015-2016 y 2016-2017 en las cuales se perdió el máximo de nidos de la temporada. La tendencia poblacional va en decremento.


\subsection*{Isla Pajaros}

\begin{figure}[H]
\hspace{-2cm}
    \includegraphics[scale=0.7]{figures/pngPopulationGrowRateCormorantPajaros.png}
\caption{En azul se muestran los máximos de la cantidad de nidos activos de cormorán orejón para cada temporada (Sep-Abr) del periodo 2012-2018 en Isla Pajaros en Sinaloa. En naranja el modelo de crecimiento poblacional ajustado.}
\end{figure}


Descartamos los puntos correspondiente a las temporadas 2015-2016 y 2016-2017 en las cuales se perdió el máximo de nidos de la temporada. La tendencia poblacional va en decremento.

\bibliography{../references/tendencia_poblacional_cormoran} 
\bibliographystyle{apalike}

\end{document}