\documentclass{article} % El documento es de tipo artículo
\usepackage[utf8]{inputenc} % Paquete que permite escribir caracteres especiales
\usepackage[spanish, es-tabla]{babel} % Paquete para cambiar "Cuadro" a "Tabla" en encabezados de tablas
\usepackage{graphicx} % Paquete para importar figuras
\usepackage{hyperref} % Paquete para agregar vínculos como enlaces
\usepackage{booktabs}
\usepackage{authblk}
\usepackage{amsmath}
\usepackage{csvsimple}
\usepackage{pythontex}
\usepackage{xcolor}
\usepackage{float}
\usepackage{caption}
\usepackage[a4paper,top=2cm,bottom=2cm,left=3cm,right=3cm,marginparwidth=1.75cm]{geometry} 

\author{Fernando Alvarez y Maritza Bello}

\title{Tendencia poblacional cormorán orejón en Isla Alcatraz \\ \begin{large} Grupo de Ecología y Conservación de Islas \end{large}}

\begin{pycode}
import json

json_file_path = 'non-tabular/jsonLambdaCormorantAlcatraz.json'
with open(json_file_path, encoding='utf8') as results_file:
    results = json.load(results_file)
\end{pycode}

\begin{document}

\maketitle

\begin{abstract}
Calculamos la tasa de crecimiento para cormorán orejón en Isla Alcatraz y se encontró que la población se mantiene estable a lo largo de los ultimos 20 años.
\end{abstract}

\section*{Metodología}
\subsection*{Modelo}
A continuación se muestra el modelo del tamaño poblacional:

\begin{equation}
N(t)=N_{0}\lambda^{t}
\end{equation}

\noindent donde $N_{0}$ representa la población inicial, $\lambda$ la tasa de crecimiento fundamental y $t$ el intervalo de tiempo. 

\subsection*{Cálculo de tasa de crecimiento}

Se utilizó la base de datos \href{https://drive.google.com/drive/u/0/folders/1aXmotwcGcZjK52USWMdlZoffaMUlI0tT}{{\color{blue}\textit{\underline{conteo\_nidos\_cormoran\_isla\_alcatraz}}}}, donde tenemos datos de cantidad de nidos activos para diferentes temporadas durante los últimos 20 años. Para calcular la tasa de crecimiento fundamental $\lambda$, tomamos los máximos de cada temporada. Consideramos una temporada como el intervalo de tiempo entre los meses de septiembre hasta abril.

\section*{Resultados}

La tasa de crecimiento $\lambda$ calculada fue \py{'%.3f'%(float(results['lambda'][0]))}. A continuación podemos ver un gráfico con los datos, y el modelo ajustado. Como podemos observar, la serie de tiempo de la cantidad de nidos activos no tiene une tendencia aparente, el valor de $\lambda$ nos indica que el tamaño de la población se mantiene estable a través del tiempo.


\begin{figure}[H]
    \includegraphics[scale=0.6]{figures/pngPopulationGrowRateCormorant.png}
\caption{En azul se muestran los máximos de la cantidad de nidos activos para cada temporada (Sep-Abr) de los ultimos 20 años en Isla Alcatraz. En naranja el modelo de crecimiento poblacional ajustado.}
\end{figure}
\end{document}